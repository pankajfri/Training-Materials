\section{Introduction to Business Analytics}

Business analytics (BA) refers to all the methods and techniques that are used by an organization to measure performance. Business analytics are made up of statistical methods that can be applied to a specific project, process or product. Business analytics can also be used to evaluate an entire company. Business analytics are performed in order to identify weaknesses in existing processes and highlight meaningful data that will help an organization prepare for future growth and challenges.

The need for good business analytics has spurred the creation of business analytics software and enterprise platforms that mine an organization’s data in order to automate some of these measures and pick out meaningful insights.

Business analytics (BA) is the iterative, methodical exploration of an organization's data, with an emphasis on statistical analysis. Business analytics is used by companies that are committed to making data-driven decisions. Data-driven companies treat their data as a corporate asset and actively look for ways to turn it into a competitive advantage. Successful business analytics depends on data quality, skilled analysts who understand the technologies and the business, and an organizational commitment to using data to gain insights that inform business decisions.

Specific types of business analytics include:
\begin{itemize}
    \item Descriptive analytics, which tracks key performance indicators (KPIs) to understand the present state of a business;
    \item Predictive analytics, which analyzes trend data to assess the likelihood of future outcomes; 
    \item Prescriptive analytics, which uses past performance to generate recommendations about how to handle similar situations in the future.
\end{itemize}

From financial services to telecommunications, from media to transportation, many industries are being digitally disrupted. New entrants with innovative business models are transforming the environment and make it more competitive. Behavior, also, evolving as customers expect the experience to be continues, cross channel and customized. To be relevant in this new environment, companies need to leverage data to understand what their customers want, and find new ways to meet their expectations.

The analytics journey of successful companies and discover how Accenture ADM collect data, find relevant insights, create actionable recommendations, and at the end of the day, deliver tangible business value.
 
To understand how to leverage business analytics to create value in the actual business environment.

, data has been used for improving the performance of many businesses since at least the 90s.

Analytics is the use of: data, information technology, statistical analysis, quantitative methods, and mathematical or computer-based models To help managers gain improved insight about their business operations and make better, fact-based decisions.

 Business Analytics The use of analytical methods: 1. Either manually or automatically 2. To derive relationships from data 3. It include the access \& reporting Analysis of data supported by software to drive business performance and decision making

 Information, Analysis And Decisions: The Basics Analysis Information Why did it happens What is Happening What is Likely to happen What should I do about you ? Descriptive Analysis Diagnostics Analytics Predictive Analytics Prescriptive Analytics Analytic excellence leads to better decisions


\section{Case Study : Complete Study of Factors Contributing to Air Pollution}

https://www.analyticsvidhya.com/blog/2016/10/complete-study-of-factors-contributing-to-air-pollution/



\section{ Case Study :  Netflix- Using Big Data to Drive Big Engagement}

Netflix is the largest internet-television network in the world
 The company’s most-watched shows are generated from recommendations, which in turn foster consumer engagement and loyalty. This is why the company is constantly working on its recommendation engines.
More than 80 per cent of the TV shows and movies people watch on Netflix are discovered through the platform’s recommendation system. That means when you think you are choosing what to watch on Netflix you are basically choosing from a number of decisions made by an algorithm.
Netflix uses machine learning, a subset of artificial intelligence, to help their algorithms “learn” without human assistance. Machine learning gives the platform the ability to automate millions of decisions based off of user activities. 