\documentclass[hyperref={bookmarks=false},aspectratio=169]{beamer}
\usepackage[utf8]{inputenc}
\usepackage{standalone}
% ---------------  Define theme and color scheme  -----------------
\usetheme[sidebarleft]{CEIT}  % 3 options: minimal, sidebarleft, sidebarright

%\setbeamertemplate{footline}[frame number]

% ------------  Information on the title page  --------------------
\title[PPT Title]
{\bfseries{PPT Title}}

\subtitle{A brief overview}

\author[TrainerName] %\& Managed]
{TrainerName\inst{1} } %\and Managed\inst{2}}

\institute[CEIT]
{
  \inst{1}
  Trainer\\
  Centre of Excellence in IT,PNG
 % \and
 % \inst{2}
 % Trainer\\
 % Centre of Excellence in IT,PNG
}

\date[CEIT, 2014]
{Centre of Excellence in IT,PNG October 2019}
%------------------------------------------------------------

%------------------------------------------------------------
%The next block of commands puts the table of contents at the 
%beginning of each section and highlights the current section:

\AtBeginSection[]
{
  \begin{frame}
    \frametitle{Table of Contents}
    \tableofcontents[currentsection]
  \end{frame}
}

%------------------------------------------------------------


\begin{document}

\frame{\titlepage}  % Creates title page

%---------   table of contents after title page  ------------
\begin{frame}
\frametitle{Table of Contents}
\tableofcontents
\end{frame}
%---------------------------------------------------------


\section{Our Courses}


\begin{frame}
\frametitle{Our Courses}

We Offer 4 PG-Diploma \&
17 Certificate Courses.  \\ \vspace{1cm}
Let us introduce our courses through use cases

{\centering Use Case :  {\Large  Smart Safety Monitor \\ }}
\vspace{1cm}
Some Features of products are
\begin{enumerate}
\item Flood, Freeze and Fire detection
\item Control Thermostat and Valve Operation
\item Raise Alert
\item Predict Maintenance of machine 
\end{enumerate}
\end{frame}

\begin{frame}
	\frametitle{Components}
	The product contains components:
	\begin{enumerate}
		\item Smart Sensor and hubs programmed to detect Freeze/Fire or Flood.
		\item To process the message from sensor and send alerts
		\item A portal (App) to control valve and Thermostat operation
		\item A view to show all parameters of the platform
		\item A model to analyze the signals of machine to predict the failure
	\end{enumerate}
\end{frame}

%---------------------------------------------------------


%---------------------------------------------------------
%\begin{frame}  % Example of the \pause command
%This slide is to test mathematical formulas \pause
%
%$$E=mc^2$$ \pause
%
%as well as the ``pause'' functionality
%\end{frame}
%---------------------------------------------------------

\section{Caltech student pranks}

%---------------------------------------------------------
%Highlighting text
\begin{frame}
\frametitle{Caltech student pranks}

This is a brief introduction of \alert{Caltech pranks}.

\begin{block}{Definition}
Prank: a practical joke or mischievous act
\end{block}

\begin{alertblock}{Axiom}
Caltech pranks are a key part of the institute's history and identity.
\end{alertblock}

\begin{examples}
See the next slide for a prank example.
\end{examples}
\end{frame}
%---------------------------------------------------------


%---------------------------------------------------------
%Two columns
\begin{frame}
\frametitle{Hollywood sign}

\begin{columns}

\column{0.45\textwidth}

\begin{figure}
    \centering
    \includegraphics[width=\columnwidth]{images/CEIT_Logo_300X300.png}
    \caption{``Hollywood is still mad about that,'' says Autumn Looijen, author of \emph{Legends of Caltech III: Techer In the Dark.} \tiny{(Photo downloaded from: http://brennen.caltech.edu/autobiography/automaster2.htm)}}
    \label{fig:hollywood_prank}
\end{figure}


\column{0.55\textwidth}
In May 1987, undergraduates from Page and Ricketts houses combined forces (and several hundred dollars) to purchase enough black and white plastic, transformed the Hollywood sign to read ``Caltech''.

\small{(Reference: http://www.admissions.caltech.edu/pranks)}

\end{columns}
\end{frame}
%---------------------------------------------------------


\end{document}