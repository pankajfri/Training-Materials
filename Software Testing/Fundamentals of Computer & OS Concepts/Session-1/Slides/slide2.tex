\begin{frame}
\frametitle{Brief Definition and History of Computers}
\begin{enumerate}
	\setcounter{enumi}{1}
\item Second generation: 1947 – 1962 - This generation of computers used transistors instead of vacuum tubes which were more reliable. In
		\begin{itemize}
			\item 1951 - the first computer for commercial use was introduced to the public Universal Automatic Computer (UNIVAC 1)
			\item 1953 - International Business Machine (IBM) 650 and 700 series. Over 100 computer programming languages were developed, computers had memory and operating systems. Storage media such as tape and disk were in use also were printers for output.
		\end{itemize}
\item Third generation: 1963 - present
		\begin{itemize}
			\item The invention of integrated circuit brought us the third generation of computers. With this invention computers became smaller, more powerful more reliable and they are able to run many different programs at the same time.
			\item 1980 - Microsoft Disk Operating System (MS-Dos) was born and in 1981 IBM introduced the personal computer (PC) for home and office use. Three years later Apple gave us the Macintosh computer with its icon driven interface and the 90s gave us Windows operating system.
		\end{itemize}
\end{enumerate}
\end{frame}